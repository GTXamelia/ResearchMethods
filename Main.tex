\documentclass[journal]{IEEEtran}
\usepackage[utf8]{inputenc}

\usepackage[backend=biber]{biblatex}
\addbibresource{main.bib}

\begin{document}
\markboth{Computing after Silicon, December 2018}%
{Computing after Silicon, November 2018}

\title{Computing after Silicon}
\author{Cian Gannon,~\IEEEmembership{Software Development (Honours),~GMIT}}

\maketitle

\begin{abstract}

Silicon is the second most commonly found element on the earth only behind oxygen. Silicon is the foundation on which we build all electronic circuits, it is an ideal semiconductor, which is great as it's extremely common too. But as we all know, every good thing must come to an end. As we keep decreasing the size of transistors silicon is starting to show issues as we are getting to sizes of 7nm and below. Currently we have 14nm in the commercial market, at this size there are only around 67 atoms per  transistor. When we get to 7nm we only have around 33 atoms and this is where we start running into problems.

\end{abstract}

\section{Introduction}
Silicon is building block of computing, but is it nearly at the end of its life? The first transistor was invented in bell labs in 1947 \cite{8896076320180101}, and ever since we've been working to decrease its size to be ever smaller than the previous iteration. We've reached a point where the current consumer chips are at 7nm (Nanometre) and with silicon atom \cite{8947487520180101} being around 210pm (Pictometre) which is 0.21nm so a silicon transistor at 7nm contains only 33 atoms per transistor. And this size keeps getting smaller with 5nm being currently under development which would only contain 24 atoms. The size of transistors is getting smaller and smaller as the years go on, but we are now at a point where the limit on there size is not only constrained by refining manufacturing process but now also by size of atoms. Silicon has been used in all modern computers and mobile phones. Silicon is the best element on earth to use in chip manufacturing for two reasons. 1. It is a great semiconductor which is necessary for modern computing and 2. It is a highly abundant element on earth being the second most common after oxygen with around 28\% of the earths mass. But even with these two properties it isn't enough. With researchers looking for alternate methods through software and hardware to improve performance. Other researchers realize the core issue is with the silicon production process we currently use is at an end, just like past technologies move to different technologies to keep up with current trends and technological improvement by changing so to do computers. Currently researchers are looking into Graphene \cite{nicol_2018} which is an allotrope of carbon consisting of a single layer of carbon atoms arranged in a hexagonal lattice. Graphene is stronger than any known material with a tensile strength of  of 130 giga-pascals. Compare Graphene to steel which has a tensile strength of around 400 mega-pascals making Graphene 3.25x stronger than steel. This makes Graphene a great alternative to silicon as its strength would help keep transistors together when dealing with transistors in the atom size range. Due to Graphene's strength it also allows for higher speeds of 100GHz \cite{johnson_2010} and above. With current processors only going as high as 5GHz before having heat issues.

\subsection{History of Silicon use}


\printbibliography

\end{document}

