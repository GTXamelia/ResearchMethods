\documentclass[journal]{IEEEtran}
\usepackage[utf8]{inputenc}

\usepackage{biblatex}[backend=biber]
\addbibresource{main.bib}

\begin{document}
\markboth{Computing after Silicon, December 2018}%
{Computing after Silicon, November 2018}

\title{Computing after Silicon}
\author{Cian Gannon,~\IEEEmembership{Software Development (Honours),~GMIT}}

\maketitle

\begin{abstract}

Silicon is the second most commonly found element on the earth only behind oxygen. Silicon is the foundation on which we build all electronic circuits, it is an ideal semiconductor, which is great as it's extremely common too. But as we all know, every good thing must come to an end. As we keep decreasing the size of transistors silicon is starting to show issues as we are getting to sizes of 7nm and below. Currently we have 14nm in the commercial market, at this size there are only around 67 atoms per transistor. When we get to 7nm we only have around 33 atoms and this is where we start running into problems.

\end{abstract}

\begin{Introduction}
Silicon is building block of computing, but is it nearly at the end of its life? The first transistor was invented in bell labs in 1947, and ever since we've been working to decrease its size to be ever smaller than the previous iteration. 
\end{Introduction}
\cite{591665}

\printbibliography

\end{document}

