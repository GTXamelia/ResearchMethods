\documentclass[journal]{IEEEtran}
\usepackage[utf8]{inputenc}

\usepackage[backend=biber]{biblatex}
\addbibresource{main.bib}

\begin{document}
\markboth{Computing after Silicon, December 2018}%
{Computing after Silicon, November 2018}

\title{Computing after Silicon}
\author{Cian Gannon,~\IEEEmembership{Software Development (Honours),~GMIT}}

\maketitle

\begin{abstract}

Silicon is the second most commonly found element on the earth only behind oxygen. Silicon is the foundation on which we build all electronic circuits, it is an ideal semiconductor, which is great as it's extremely common too. But as we all know, every good thing must come to an end. As we keep decreasing the size of transistors silicon is starting to show issues as we are getting to sizes of 7nm and below. Currently we have 14nm in the commercial market, at this size there are only around 67 atoms per  transistor. When we get to 7nm we only have around 33 atoms and this is where we start running into problems.

\end{abstract}

\section{Introduction}
Silicon is building block of computing, but is it nearly at the end of its life? The first transistor was invented in bell labs in 1947 \cite{8896076320180101}, and ever since we've been working to decrease its size to be ever smaller than the previous iteration. We've reached a point where the current consumer chips are at 7nm (Nanometre) and with silicon atom \cite{8947487520180101} being around 210pm (Pictometre) which is 0.21nm so a silicon transistor at 7nm contains only 33 atoms per transistor. And this size keeps getting smaller with 5nm being currently under development which would only contain 24 atoms. The size of transistors is getting smaller and smaller as the years go on, but we are now at a point where the limit on there size is not only constrained by refining manufacturing process but now also by size of atoms. Silicon has been used in all modern computers and mobile phones. Silicon is the best element on earth to use in chip manufacturing for two reasons. 1. It is a great semiconductor which is necessary for modern computing and 2. It is a highly abundant element on earth being the second most common after oxygen with around 28\% of the earths mass. But even with these two properties it isn't enough. With researchers looking for alternate methods through software and hardware to improve performance. Other researchers realize the core issue is with the silicon production process we currently use is at an end, just like past technologies move to different technologies to keep up with current trends and technological improvement by changing so to do computers. Currently researchers are looking into Graphene \cite{nicol_2018} which is an allotrope of carbon consisting of a single layer of carbon atoms arranged in a hexagonal lattice. Graphene is stronger than any known material with a tensile strength of  of 130 giga-pascals. Compare Graphene to steel which has a tensile strength of around 400 mega-pascals making Graphene 3.25x stronger than steel. This makes Graphene a great alternative to silicon as its strength would help keep transistors together when dealing with transistors in the atom size range. Due to Graphene's strength it also allows for higher speeds of 100GHz \cite{johnson_2010} and above. With current processors only going as high as 5GHz before having heat issues.

\subsection{History of Silicon Use}
In the early days of computing vacuum tubes \cite{8931727020180101} were used before in what were at the time super computers which were highly specialized. Most consumer devices with transistors were basic car radios which were still a major improvement over vacuum tube radios which was cheaper car radio and more efficient car radio as transistors 'bled' less heat.

A year after the introduction of the transistor in car radios in 1956 Bell Labs \cite{8896076320180101} developed a silicon transistor which was the beginning of what would become the basis of modern computing. This transistor was very basic using molten silicon.

In 1971 Intel developed the first silicon processor called the Intel 4004. Although it is very primitive in today's standards it was revolutionary. It was a 4 bit processor with a clock speed 740 kHz. The Intel 4004 only had 2300 transistors, compare this to the latest Intel i9 processor which contains somewhere around 100 million transistors per square millimeter, a big leap from their first microprocessor. 

As time went on and multiple companies started developing more and more transistors the price fell and so did the process of manufacturing. Silicon is a great chose as a semiconductor as its abundant and once manufacturers such as Intel started developing silicon transistors the price fell and the density increased just as Gordon Moore predicted it would. This created as some call it the 'Computer Age' (Also called the information age). An era where computers went from machines where users would have to be trained to operate them like any other machine of the time, to devices which everyone would own and use on a daily basis.

\subsection{Moore's Law}
Moore's law \cite{12560013020170101} is more of a prediction made by Intel co-founder Gordon Moore. It states that the number of transistors in a dense integrated circuit doubles about every two years and the cost decreases by half. This observation was made in 1965 and has been true since it was made. Although the first time since it was made we are now seeing it come to and end. We are getting to a point where atoms are getting in the way of making transistors smaller and we have to rethink how we produce microprocessors \cite{591665}.

\subsection{Current progress with silicon}
Today's silicon process is a more refined process than we used at the beginning of the development of microprocessors in 1971. We have refined the process to get around 2 billion to 3 billion transistors in one microprocessor. which is a big leap from the 2300 found in the first microprocessor.

Even with all these refinements we've slowed down in the last few years. Intel is staying with their 14nm manufacturing process while they develop a process to create a 10nm microprocessor. Intel have been using 14nm manufacturing process since 2015. Intel is expected to fully release their 10nm design in 2019 a full four years after releasing their 14nm line of microprocessors. This type of behaviour is expected as we make smaller and smaller transistors with transistors smaller than 7nm requiring new methods due to quantum tunneling. It is once we hit 5nm where experts from 10 years ago presumed it was the end of Moore's law. \cite{591665}

\section{Current Silicon Development}
As of 2018 transistors have gotten as small as 7nm with low powered machines such as the iPhone X. Intel and AMD (Microprocessor manufacturers) are both rumored to release new microprocessors under the current 14nm technology. Although this change has been long overdue with both manufacturers using the 14nm process for years (Intel since 2014). Intel has repeatedly delayed their 10nm chips \cite{dent_2018} which has caused investors to question the current technology. Creating a stable silicon chip with transistors as small as 7nm requires a lot of challenges to be addressed and fixed before consumer roll-out. Intel are facing a much bigger issue than the small iPhone X such has keeping all current technology such as its Hyper-Threading Technology which are required by one of their biggest market bases, the cloud computing sector. 

\subsection{Quantum Tunneling}
Quantum tunneling is a complex quantum mechanical phenomenon \cite{70394920140101} but to save on time and limit it's scope to the silicon transistor it is the theory that particles that shouldn't go through a material do. The flaw with silicon transistors is that at 5nm and below this becomes a huge problem. Silicon transistors at below 7nm sit so close together that the effect called quantum tunneling occurs. This effect ruins computers as transistors cannot be reliably turned off and on and may just get stuck in an on position.

\subsection{Status of Moore's Law}
Moore's law has been correct since its inception, but only now are we reaching the limit of the size of a transistor. \cite{591665} In the last 5 years production of smaller transistors have taken more time then previous developments. This is due to the increased complexity in developing a new production technique that will be as reliable as current techniques. Intel have yet to release a new transistor since 2014 with their 14nm process. And even with the delay from this year to next year the transistor will only be 10nm in size. So not only did it take them 5 years to create a new manufacturing process but it is not half the current size. Intel have been in the microprocessor business since the start with the worlds first microprocessor 'Intel 4004'. And as we get smaller and smaller transistors we can see old businesses like Intel struggle to get a new process out in time.

\subsection{Silicon Lottery}
All chips as not made equally, in enthusiast groups who look for the best chips on the market there is a term used by them called the 'Silicon Lottery'. The term silicon lottery is used to describe the way microprocessors are manufactured. All chips that are manufactured run at a standard clock speed such as 4.2GHz for AMD's FX-8350 which mean transistors operate 4.2 billion times a second. But this is the manufacturing speed that all of the FX-8350 chips will run but due to each silicon dye being of very slight different quality they can be over clocked to provide higher clock speeds such as 4.5GHz. The problem is some chips may operate at that speed or higher, some may not allow for much over clocking at all. All silicon chips come with differing quality, in most cases this does not matter to the average consumer but to the enthusiast groups around the world getting the highest quality silicon chip can allow them to run at speed of up to 5GHz on very old chips. The silicon lottery shows how we already have trouble creating very standard chips that are all alike.

\section{Proposed Alternatives}
With the end of Moore's law drawing ever closer researchers are looking into different ways to keep current progress going at the same rate as the last 50 years. There are many different solutions being talked about some software, some hardware. The hardware application are what will keep progress going. The software solutions just postpone the inevitable problem with silicon manufacturing process.

\subsection{Silicon-Germanium Mix}
Silicon-Germanium mix is an short term solution to the silicon transistor problem \cite{8947469320180101}. Germanium shares many of its properties with Silicon. They are both Metalloids, they both chemically similar to one another. The main reasons to add Germanium to the silicon transistor would be to solve the issue of quantum tunnelling through the oxide layer of the gate, which at transistors below 7nm would cause huge problems with machine performance. Germanium can be used on top of the traditional silicon chips as a bridge between gates to solve quantum tunneling \cite{ye_2016}. We know Germanium works as early on germanium was used in transistors in the 50's but was later replaced by the cheaper and more abundant silicon semiconductor. 

\subsection{Graphene}
Graphene is the long term solution for fixing the silicon issue with transistors. Graphene is the strongest known material, much stronger than silicon and even stronger than steel. Graphene would not completely replace silicon, much like the Silicon-Germanium solution graphene would be placed on a silicon chip to reduce costs. Graphene would be used in the essential parts of the computer like the central processing unit before being introduced later into other components \cite{48600520130101}.

Graphene is being proposed due to its strength as higher loads. The fastest graphene transistor is about ten times faster than the fastest silicon alternative \cite{bourzac_2010}. Although users won't see near that kind of speed in commercial use, they will see a percentage of it and that is still much faster than our traditional silicon based transistors. Due to graphene's strength it is able to operate at a much higher rate than silicon transistors at a rate of 100GHz (Gigahertz) which is the transistors operating 100 Billion times a second much faster than the silicon transistors of today. 

\section{Development by Companies}
Every major chip manufacturer is trying to solve this issue with workarounds of all kinds. This is an issue that all manufacturers know they will need to solve eventually so they are all trying to figure ways to fix it or have a workaround it until they fix it.

\subsection{Intel}
Intel is the first company to release a commercially available microprocessor back in 1971 with their 'Intel 4004' microprocessor. Ever since 1971 Intel have been at the forefront of microprocessor development have multiple plants worldwide and multiple R\&D site worldwide in order to keep their prevalence on the market. Intel have the biggest market share of desktop microprocessors with Steam (a computer gaming platform with over 150 Million active) having 83\% of its users using an Intel microprocessor with its only competitor having 17\%. Intel would like to keep this market share the same so to ensure that they do they have multiple teams working on developing transistors as small as 3nm. How they plan on tackling the silicon issue is unknown.

\subsection{AMD}

\subsection{Mobile}


\printbibliography

\end{document}

